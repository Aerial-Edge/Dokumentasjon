\section{Blob Detection}

To identify the desired objects, our blob detection algorithm utilizes the HSV color space, which separates the hue, saturation, and value/brightness components of an image. By selecting appropriate color ranges, we are able to isolate specific objects based on their distinctive color properties. Once we have found a colored object through color-based detection, our algorithm employs mathematical equations to analyze the contour properties of these objects. Specifically, we determine whether the contours of the detected blobs exhibit circular characteristics. This step helps differentiate the desired balls from other shapes or artifacts in the image. 

Blob detection offers a high level of modifiability, allowing it to be tailored to meet specific computational requirements. By adjusting key parameters, the algorithm can be implemented with minimal computational resources, making it suitable for applications where processing power is limited. Alternatively, a larger set of parameters can be utilized, enabling more intricate analysis and detection of complex blobs. However, it should be noted that this approach requires higher computational resources. 

Considering the limitations of the Raspberry Pi 4 without acceleration hardware, the modifiability of blob detection allows us to strike a balance between detection accuracy and computational efficiency. We can optimize the algorithm by fine-tuning parameters to ensure reliable performance on this hardware platform. 







